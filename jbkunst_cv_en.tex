% LaTeX resume using res.cls
\documentclass[margin]{res}
\usepackage[utf8]{inputenc}
\setlength{\textwidth}{13cm} % set width of text portion

% IWONA font
\usepackage[light,condensed,math]{iwona}
%\usepackage[T1]{fontenc}

\begin{document}

\moveleft.5\hoffset\centerline{\large\bf Joshua Benjamín Kunst Fuentes}

\vspace{.2cm}

\moveleft.5\hoffset\centerline{jbkunst@gmail.com  $\diamond$ http://jkunst.com/ $\diamond$ (+569) 842-370-64}

\moveleft\hoffset\vbox{\hrule width\resumewidth height 1pt}

\vspace{.5cm}

\moveleft.5\hoffset\vbox{\textit{Experiencia en la aplicación de metodologías Estadístcias y de Data mining para la modelación y predicción de información. Habilidades en la creación, evaluación, y monitoreo de modelos predictivos. Interés y conocimiento en el desarrollo de software y en la automatización de procesos.}}

\vspace{.2cm}

\moveleft\hoffset\vbox{\hrule width\resumewidth height 1pt}\vspace{.5cm}

\begin{resume}

\section{Education}
    {\sl Magíster en Estadística}\hfill 2009\\
    Facultad de Matemáticas, Pontificia Universidad Católica.

    {\sl Estadístico}\hfill 2008\\
    Facultad de Matemáticas, Pontificia Universidad Catlólica.

    {\sl Licenciado en Matemáticas}\hfill 2007\\
    Facultad de Matemáticas, Pontificia Universidad Cataólica.

\vspace{1cm}

\section{Experience}

    {\sl Senior Analytics}\hfill Ene. 2013 a la fecha\\
    Anaytics, Foris.

    {\sl Analista de Riesgo Sénior}\hfill Feb. 2011 a Ene. 2013\\
    Technical Analytics Service, Equifax Chile.

    {\sl Analista Scoring}\hfill Oct. 2010 - Feb. 2011\\
    Gerencia División Riesgo Crédito, CorpBanca.

    {\sl Profesor del curso Estadística I}\hfill 2010\\
    Ingeniería Comercial, Universidad Adolfo Ibañez.

\vspace{1cm}

\section{Skills}
    {\sl Software Estadístico}: Usuario avanzado en R; SPSS, Clementine, SAS, STATA, Minitab.

    {\sl Lenguajes de Programación}: Python, Java.
    
    {\sl Frameworks Web}: Django.

    {\sl Base de datos}: T-SQL, MySQL, PostgreSQL.

    {\sl Automatización de Reportes}: Knitr, Sweave, RMarkdown.

    {\sl Otras Aplicaciones}: RStudio, Latex, Maple, Matlab, Microsoft Office.

\vspace{1cm}

\newpage

\section{Seminarios y\\Actividades}
    
    {\sl Participante en International Space Apps Challenge – Chile}\hfill Abr. 2013\\
    Proyecto PDSR, https://github.com/gvegayon/PDSR.
    
    {\sl Organizador Grupo de Usuarios de R en Chile}\hfill Mar. 2013\\
    Meetup de Usuarios, http://www.meetup.com/useRchile/.
    

    {\sl Fair Isaac Company}\hfill Dic. 2010\\
    Construya y aplique modelos en menos tiempo con Model Builder 7.1.

    {\sl VIII Congreso Latinoamericano de Sociedades de Estadística\hfill Oct. 2008\\
    Instituto de Estadística, Facultad de Ciencias Económicas y de Administración de la Universidad de la Repúblicica, Uruguay.

    {\sl XXXIV Jornadas Nacionales de Estadística\hfill Oct. 2007\\
    Pontificia  Universidad Católica de Valparaíso, Asistente.

    %\{\sl XXXIV Semana de la Matemática\hfill Oct. 2006\\Pontificia  Universidad Católicade Valparaíso, Asistente.

    {\sl XV Olimpiada Nacional de Matemáticas\hfill Oct. 2003\\
    Participante Finalista.

    {\sl Introducción a la Programación\hfill Ene. 2003\\
    Escuela de Ingeniería y Ciencias, Universidad de Chile.

    {\sl Profesor Auxiliar Laboratorio de Probabilidades}\hfill 2009\\
    Facultad de Economía y Administracíon, Univerisdad Gabriela Mistral.

    %\{\sl Ayudante Cátedra Introducción a la Estadística}\hfill 2008 - 2009\\ Facultad de Matemáticas, Pontificia Universidad Católica

    {\sl Ayudante Cátedra Métodos Bayesianos}\hfill 2007 - 2009\\
    Facultad de Matemáticas, Pontificia Universidad Católica

    {\sl Ayudante Cátedra Procesos EstocásticosAplicados}\hfill 2007\\
    Facultad de Matemáticas Pontificia Universidad Católica

\vspace{1cm}

\section{Profiles}
    
    {\sl Github}\hfill http://github.com/jbkunst\\
    
    {\sl LinkedIn}\hfill http://www.linkedin.com/in/joshuakunst

\vspace{1cm}


\end{resume}

\vspace{0.5cm}

\hfill {\sl \today}

\end{document}
